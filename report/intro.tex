\chapter{Introduction}
\label{cha:intro}

Zoek hier een bron die een quote maakt over zelfrijdende wagens. Inspiratie in thesissen en taak voertuigsystemen (iemand die er veel van weet)

\section{Wat moet er vermeld worden}
Goed refereren naar de bronnen waar over spreekt. Bespreek de vooruitgangen die gemaakt zijn in het gebruik van zelf rijden wagens en probeer te verkopen wrm het nu nuttig is om naast veiligheid te kijken naar comfort. 

wagenziektes verminderen
een klant koopt een auto waar hij zich goed in voelt --> comfort is specifiek
comfort = vertrouwen




(meer op het einde van de introductie)
Dit hoofdstuk geeft vooral een overzicht over de theorie en de opbouw van het probleem.
Wat zijn de doelen van het project --> zie slides / wat is de probleemstelling.
Wat is de kapstok --> drie stappen learning, planning, controlling, validatie
Hoe zal het er in grote lijnen uit gaan zien en welke softwares zijn er allemaal aan te pas gekomen
Welke voorzieningen heeft Siemens hieromtrent.

Wat is een optimal control problem en wat is model predictive control? (zie paper VS)
--> wat dieper ingaan op wat ik geleerd heb bij het vak optimization --> zie ook important thesis notes van optimization

Maak een overview van het control sequence --> flowchart zoals op p5 zwitser. Maak zo veel mogelijk om de stromingen van de stappen bijvoordbeeld 3 stappen in project goed weer te geven. Veel figuren!


%%% Local Variables: 
%%% mode: latex
%%% TeX-master: "thesis"
%%% End: 
