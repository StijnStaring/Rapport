\chapter{Introduction}
\label{cha:intro}
"Society expects autonomous vehicles to be held to a higher standard than human drivers." \cite{Prof.Amnon} This quote is setting the tone of the technology in autonomous driving. In order to be accepted to the public autonomous vehicles should perform as least as good as the conventional human driver on parameters as safety. Despite widespread research on self-driving vehicles the acceptance by the user stays only limited.\cite{Bae2019} The purchase behaviour of customers is naturally be linked with a good feeling which is directly connected with comfort. Also in order to gain more trust by the public it is necessary to look to this aspect. This immediately leads to the questions what comfort during driving exactly is and how to measure it.\\

Driving comfort is a personal experience and also depend on the emotional state that the driver(s) are in. This means that more than one driving style for autonomous vehicle-driving should be identified. The state of the driver can be manually communicated with the vehicle. \cite{Eindhoven2019} In order to identify the specific comfort preferences of the driver, it should be able to learn them by demonstration. \cite{Kuderer2015a}\\
Despite that each driver has its own preferences it should be noted that they are based on a common notion of comfort where different trade-offs are made. For example some drivers prefer a more aggressive than others which manifests itself in more brusque accelerations but both aggressive and defensive drivers can be evaluated by the same comfort criteria. This will later be translated into a comfort objective where different weights are used in order to quantify different comfort trade-offs made. \\

In order find comfort criteria which can be used to distinguish different drivers research about the notion of comfort is necessary. Research based on passenger surveys has been conducted on car sickness in autonomous vehicles in public road transport \cite{Turner1999} where was concluded that lateral motions and non-smooth driving behaviour could be best correlated with car sickness. 



Despite widespread research on self-driving, user acceptance remains an essential part of successful market penetration; this forms the motivation behind studies on human factors associated

\section{Wat moet er vermeld worden}
Goed refereren naar de bronnen waar over spreekt. Bespreek de vooruitgangen die gemaakt zijn in het gebruik van zelf rijden wagens en probeer te verkopen wrm het nu nuttig is om naast veiligheid te kijken naar comfort. 

wagenziektes verminderen
een klant koopt een auto waar hij zich goed in voelt --> comfort is specifiek
comfort = vertrouwen




(meer op het einde van de introductie)
Dit hoofdstuk geeft vooral een overzicht over de theorie en de opbouw van het probleem.
Wat zijn de doelen van het project --> zie slides / wat is de probleemstelling.
Wat is de kapstok --> drie stappen learning, planning, controlling, validatie
Hoe zal het er in grote lijnen uit gaan zien en welke softwares zijn er allemaal aan te pas gekomen
Welke voorzieningen heeft Siemens hieromtrent.

Wat is een optimal control problem en wat is model predictive control? (zie paper VS)
--> wat dieper ingaan op wat ik geleerd heb bij het vak optimization --> zie ook important thesis notes van optimization

Maak een overview van het control sequence --> flowchart zoals op p5 zwitser. Maak zo veel mogelijk om de stromingen van de stappen bijvoordbeeld 3 stappen in project goed weer te geven. Veel figuren!


%%% Local Variables: 
%%% mode: latex
%%% TeX-master: "thesis"
%%% End: 
