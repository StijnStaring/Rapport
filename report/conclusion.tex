\chapter{Conclusion}
\label{cha:conclusion}
The contribution made with this thesis is the development of an algorithm based on inverse optimal control, that can learn driver specific experiences of comfort during a lane change captured in weighting factors of an objective function.\\
First a literature study was conducted in order to identify comfort during driving. Here it was found that smooth paths play an important role in contributing to the retrieved amount of comfort of a human driver. Next, it was looked at the theoretical idea of feature-based inverse optimal control. This was the starting point of this thesis and the goal was to implement this theoretical idea for in practise useable models e.g. $15$ dof Amesim model.\\
In the following chapter, learning from ideal data was discussed. Here it was presented how ideal data was generated and validated and it was shown that the developed algorithm was able to accurately learn the chosen lateral weighting factors that define the lane change maneuver. Thereby it was displayed that the estimation of the gradient $\pdv{\bm{F}_{diff}}{\bm{\theta}}$ by $ \bm{F_{obs}} - \bm{F}(\bm{r}_{expected})$ can be used in order to converge towards the data feature vectors used.\\
Furthermore, learning with a $15$ degrees of freedom vehicle model was analysed. For this a tracking MPC formulation was developed and it was shown that also here 
the algorithm was able to accurately learn the chosen lateral weighting factors while the learned feature values converge towards the data feature values.\\
As last the theoretical idea of an improved method of updating the weighting factor vector $\bm{\theta}$ during the learning iterations, was discussed.\\

This work has its practical usage in the avoidance of manual tuning of a comfort objective which when identified, can be used for path planning. Because this research is conducted together with Siemens, it is in line with their research and this thesis can be used to build further on.\\
The maneuver discussed is a lane change maneuver. The only way that this is defined in Eq. \ref{opt:basic_opti_w} is by setting constraints on the end state Eq. (\ref{eq:H}), so that a lane change is performed. It is thereby possible to change these end constraints so that also other maneuvers are learned e.g. an acceleration maneuver.\\
During this thesis the environment is not assessed although this is a necessary condition in order to take comfort into account as followed out of section \ref{s:comfort_parameters}. As discussed in section \ref{s:obj}, it is possible to add features that encapsulate notions about the environment of the autonomous vehicle so that the feeling of driving safe can be assured. Therefore a suggestion for future work is to perform the developed algorithms on real driver data which also contains information about the vehicle its surroundings and to further validate the theoretical weighting factor update concept, proposed in chapter \ref{cha:Enhancement}.

%
%The final chapter contains the overall conclusion. It also contains
%suggestions for future work and industrial applications.
%Application: say something about hierarchical control. The comfort controller can work as an inferior controller of the safety of the vehicle. 



%%% Local Variables: 
%%% mode: latex
%%% TeX-master: "thesis"
%%% End: 
