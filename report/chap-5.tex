\chapter{Validation with expert driver data}
\label{cha:Validation}

Bespreek de validatie van de methode. Implementeer similaties in prescan.
Bespreek de verschillende software tools bij Siemens --> Amesim, simulink, prescan.
Hoe werken ze samen en hoe wordt de validatie precies gedaan? Wat zijn de resultaten?

Install amesim and write a chapter about how the dataset is generated. How is the amesim model defined etc. 

how accurate results are learned when the assumption on the observations is violated. (assumping generated by a comfort cost function)

The objective function of the MPC is slightly different of the one used in the learning algorithm. In the training set it was important to explain the observed data but it is known beforehand that the driver demonstrated a feasible path and wants to do a lane change. Now the situation is different and features that assess the environment are a necessity. For this reason the objective function of the learning algorithm is extended with features that handle collision avoidance and following distance. (Lane change behavior is influenced by the environment! )



%\begin{algorithm}[H]
%	\SetAlgoLined
%	\KwResult{$\bm{\theta}_{opti}$ }
%	initialization: \\
%	$\bm{\theta} = [1,1,...,1,1] $\\
%	$\bm{\tilde{f}} = \frac{1}{m} \sum_{i=1}^{m}\bm{f}(\bm{r}_{obs,i}) $ \;
%	\While{Not converged}{
%		\For{i = 1...m}{	
%			$\smash{\displaystyle\min_{r_{exp}}} \hspace{3 mm} \bm{\theta}^T\cdot\bm{f}(\bm{r}_{exp})$ \;
%			\vspace{2 mm} 
%			constraints: \\
%			\hspace{5 mm}$vehicle\_model$\;
%			\hspace{5 mm}$begin = initial(observed_i)$\;
%			\hspace{5 mm}$end = [y = lane\_distance_{i}, vy = 0, ay = 0, jy = 0]$\;
%			\hspace{5 mm}$physical\_boundaries\_vehicle$\;		
%		} \\
%		
%		$\bm{f}_{obtained} = \frac{1}{m} \sum_{i=1}^{m}\bm{f}(\bm{r}_{exp,i})$ \;
%		$\Delta \bm{\theta} = \alpha\cdot(\bm{f}./\bm{\tilde{f}}-\bm{1})$\;
%		$\bm{\theta} = \bm{\theta} + \Delta \bm{\theta}$\;
%		\If{$\Delta \bm{\theta} \leq \epsilon$}{
%			$return \hspace{1 mm}\bm{\theta}$\;		
%		}
%	}
%	\caption{\textbf{Learning algorithm}}
%\end{algorithm}




















\section{The First Topic of this Chapter}
\subsection{Item 1}
\subsubsection{Sub-item 1}


\subsubsection{Sub-item 2}


\subsection{Item 2}


\section{The Second Topic}


\section{Conclusion}

%%% Local Variables: 
%%% mode: latex
%%% TeX-master: "thesis"
%%% End: 
