\chapter{The Learning Algorithm\\}
\label{cha:2}
Herinner de lezer nog even de structuur die gaat worden gevolgd. 
Learning, planning, tracking, validatie.
Dit hoofdstuk zal over het learning gaan.
Hoe is algorithm opgebouwd? Wrm wordt dit zo gedaan?
Welke vehicle modellen wordt er gebruikt? Wrm mag men hier een simple vehicle mode gebruiken?
Dit is gemachtigd omdat men hier de omgeving wil scannen voor een feasible pad --> dit wordt trager gedaan dan de tracking.(tracking zal gebruik maken van een meer complex model) Path planning ligt focus vooral op de omgeving.
Goed refereren naar het rapport en VS rapport
Wat zijn de assumpties die werden genomen?

Hoe zal de methode gevalideerd worden? Leg de twee methodes uit: code generatie en kijken of de wegings factoren terug gevonden kunnen worden? Mappen de feature values met de values van het geobserveerde pad? --> is het doel dat gevolgd probeert te worden haalbaar? 

Ga hier niet meer te diep in op de entropie. Leg het hier meer intuitief uit om de lezer niet te verwaren. 

Vermeld afleiding van algortihm. Leg uit in Thesis hoe komt aan gradient die gebruikt. Zie papers: Ziebart et al and Kretzschmar et al.

Modeleer een andere bestuurder. Can try to reproduce a data set with a change of parameters which represents a different driver. Can check that the learned model is also different. Hiermee aantonen dat er ook echt andere wegingsfactoren worden gegenereerd en dat de specifieke driving characteristics worden meegenomen.

Ligt een tipje van de sluier op : hoe zal de data gegenereerd worden? (dit moet kort blijven)
Plot simulink model en duidt de blokken aan die zullen worden ingevuld. Hier gaat dieper in gegaan worden in de volgende hoofdstukken. 

Maak een vermelding dat men het menselijke gedrag van het geleerde model kan nagaan met een Turing test.\\

Maak een plotje zoals paper Learning to Predict Trajectories of Cooperatively Navigating Agents --> feature variance afwijking en average error. (zelfde plotjes als al de papers)\\



Check uitgebreide samenvatting van hoofdpaper op oneNote.

Schrijf een paragraaf over de theta update --> zie RPROP methode --> beschrijf wrm beter is dan andere methodes die gezien werden. Bespreek hoe de parameters werden gekozen. 


Kan vermelding maken dat in deze thesis de features zijn gekozen met de hand --> men kan proberen om de features ook te leren van date (Characterizing Driving Styles with Deep Learning)

\clearpage


%Afleiding van exponentiël functie zie paper: Feature-based prediction of trajectories for socially compliant navigation (foto) --> weights are lagrange coefficients.


\section{The First Topic of this Chapter}


%\subsection{An item}
%A master's thesis is never an isolated work. This means that your text must
%contain references. On-line documents\cite{wiki} as well as
%books\cite{pratchett06:_good_omens} can be referenced.
%
%\section{Figures}
%Figures are used to add illustrations to the text. The \fref{fig:logo} shows
%the KU~Leuven logo as an illustration.
%\begin{figure}
%  \centering
%  \includegraphics{logokul}
%  \caption{The KU~Leuven logo.}
%  \label{fig:logo}
%\end{figure}
%
%\section{Tables}
%Tables are used to present data neatly arranged. A table is normally
%not a spreadsheet! Compare \tref{tab:wrong} en \tref{tab:ok}: which table do
%you prefer?
%
%\begin{table}
%  \centering
%  \begin{tabular}{||l|lr||} \hline
%    gnats     & gram      & \$13.65 \\ \cline{2-3}
%              & each      & .01 \\ \hline
%    gnu       & stuffed   & 92.50 \\ \cline{1-1} \cline{3-3}
%    emu       &           & 33.33 \\ \hline
%    armadillo & frozen    & 8.99 \\ \hline
%  \end{tabular}
%  \caption{A table with the wrong layout.}
%  \label{tab:wrong}
%\end{table}
%
%\begin{table}
%  \centering
%  \begin{tabular}{@{}llr@{}} \toprule
%    \multicolumn{2}{c}{Item} \\ \cmidrule(r){1-2}
%    Animal    & Description & Price (\$)\\ \midrule
%    Gnat      & per gram    & 13.65 \\
%              & each        & 0.01 \\
%    Gnu       & stuffed     & 92.50 \\
%    Emu       & stuffed     & 33.33 \\
%    Armadillo & frozen      & 8.99 \\ \bottomrule
%  \end{tabular}
%  \caption{A table with the correct layout.}
%  \label{tab:ok}
%\end{table}
%
%
%\section{Conclusion}
%The final section of the chapter gives an overview of the important results
%of this chapter. This implies that the introductory chapter and the
%concluding chapter don't need a conclusion.



%%% Local Variables: 
%%% mode: latex
%%% TeX-master: "thesis"
%%% End: 
