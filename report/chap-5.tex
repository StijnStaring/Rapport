\chapter{Future work}
\label{cha:Validation}

Bespreek de validatie van de methode. Implementeer similaties in prescan.
Bespreek de verschillende software tools bij Siemens --> Amesim, simulink, prescan.
Hoe werken ze samen en hoe wordt de validatie precies gedaan? Wat zijn de resultaten?

\section{The First Topic of this Chapter}
\subsection{Item 1}
\subsubsection{Sub-item 1}


\subsubsection{Sub-item 2}


\subsection{Item 2}


\section{The Second Topic}


\section{Conclusion}

%\begin{algorithm}[H]
%	\SetAlgoLined
%	\KwResult{$\bm{\theta}_{opti}$ }
%	initialization: \\
%	$\bm{\theta} = [1,1,...,1,1] $\\
%	$\bm{\tilde{f}} = \frac{1}{m} \sum_{i=1}^{m}\bm{f}(\bm{r}_{obs,i}) $ \;
%	\While{Not converged}{
%		\For{i = 1...m}{	
%			$\smash{\displaystyle\min_{r_{exp}}} \hspace{3 mm} \bm{\theta}^T\cdot\bm{f}(\bm{r}_{exp})$ \;
%			\vspace{2 mm} 
%			constraints: \\
%			\hspace{5 mm}$vehicle\_model$\;
%			\hspace{5 mm}$begin = initial(observed_i)$\;
%			\hspace{5 mm}$end = [y = lane\_distance_{i}, vy = 0, ay = 0, jy = 0]$\;
%			\hspace{5 mm}$physical\_boundaries\_vehicle$\;		
%		} \\
%		
%		$\bm{f}_{obtained} = \frac{1}{m} \sum_{i=1}^{m}\bm{f}(\bm{r}_{exp,i})$ \;
%		$\Delta \bm{\theta} = \alpha\cdot(\bm{f}./\bm{\tilde{f}}-\bm{1})$\;
%		$\bm{\theta} = \bm{\theta} + \Delta \bm{\theta}$\;
%		\If{$\Delta \bm{\theta} \leq \epsilon$}{
%			$return \hspace{1 mm}\bm{\theta}$\;		
%		}
%	}
%	\caption{\textbf{Learning algorithm}}
%\end{algorithm}






















%%% Local Variables: 
%%% mode: latex
%%% TeX-master: "thesis"
%%% End: 
